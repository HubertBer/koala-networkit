\section{X-fast trie}
The X-fast trie is a data structure introduced by Dan Willard in his 1983 paper \emph{Log-logarithmic worst-case range queries are possible in space \( \Theta(N) \)}~\cite{Willard1983}. For a universe of size \( M \), it supports \texttt{search}, \texttt{predecessor}, and \texttt{successor} queries in \( O(\log \log M) \) time, while insertions and deletions require \( O(\log M) \) time. The data structure uses \( O(n \log M) \) space, where \( n \) is the number of stored elements. X-fast tries were developed as a foundation for  the more advanced Y-fast tries, which incorporate X-fast tries as components.

The X-fast trie builds on the structure of a standard trie, enhancing it with binary search across trie levels to improve query efficiency. Although the \texttt{XFastTrie} class implements the predefined \texttt{PriorityQueue} interface, it must also support public predecessor queries and arbitrary key removal, as these operations are essential for integration into the Y-fast trie.

\subsection{Structure and complexity}
The core idea is to store all elements in a binary trie of height \( \lceil \log_2 M \rceil \), where the keys are treated as binary numbers, padded with zeros to ensure uniform length. Internal nodes with a missing child preserve jump pointers: if a node has no left child, it stores a pointer to the minimum leaf in its right subtree; if it has no right child, it stores a pointer to the maximum leaf in its left subtree. Only the leaves contain actual keys, and these are connected into a doubly linked list to enable constant-time access to predecessor and successor. Each node in the trie corresponds to a prefix of some stored key, and every level of the trie is augmented with a hash table that maps prefixes to their corresponding nodes.

The simplified class outline looks like this:
\begin{minted}{cpp}
class XFastTrie : public PriorityQueue<Key, Compare> {
    Key universeSize;
    int bitWidth;
    size_t size = 0;
    std::shared_ptr<Node> root;
    std::shared_ptr<Node> dummy;
    std::vector<std::unordered_map<Key, std::shared_ptr<Node>>> prefixLevels;
};
\end{minted}

The \texttt{dummy} is a sentinel node, which demarcates the beginning and end of our doubly linked list of leaves. The element right after the sentinel will be the minimum element, while the element right before it will be the maximum.

The trie consists of nodes structured as follows:

\begin{minted}{cpp}
struct XFastTrie::Node {
    std::shared_ptr<Node> children[2];
    std::shared_ptr<Node> parent;
    std::shared_ptr<Node> jump;
    std::shared_ptr<Node> linkedNodes[2];
    std::optional<Key> key;
};
\end{minted}

\begin{lemma}
The space complexity of a X-fast trie storing n keys over a universe of size \( M \) is \( O(n \log M) \).
\end{lemma}

\begin{proof}
Each node contains a constant number of pointers to other nodes and a single key. Therefore, the space requirement for a single node is constant. For each key, we maintain \( O(\log M) \) nodes. Therefore, the space requirement for the entire bitwise trie part of the structure, including the doubly linked list of leaves, is \( O(n \log M) \).
Moreover, each node is also pointed to in the \texttt{prefixLevels} structure, which therefore requires an additional \( O(n \log M) \) pointers to nodes.  
Finally, all remaining variables require constant space.
\end{proof}

We also define some helper constants to clarify referencing the left and right child (in the case of the \texttt{children} array) and the previous or next key (in the case of the \texttt{linkedNodes} array).

\begin{minted}{cpp}
static constexpr int LEFT = 0;
static constexpr int RIGHT = 1;
static constexpr int PREV = 0;
static constexpr int NEXT = 1;
\end{minted}

The \texttt{XFastTrie} is initialized by providing the universe size M:

\begin{minted}{cpp}
XFastTrie::XFastTrie(Key universeSize)
    : universeSize(universeSize),
      bitWidth(calculateBitWidth(universeSize)),
      root(std::make_shared<Node>()),
      dummy(std::make_shared<Node>()),
      prefixLevels(bitWidth + 1) {
    dummy->linkedNodes[PREV] = dummy;
    dummy->linkedNodes[NEXT] = dummy;
}
\end{minted}


\subsection{Push operation}
The \texttt{push} operation begins by traversing down the trie along the path corresponding to the \texttt{key} to be inserted. If the entire path already exists, we return \texttt{false}, indicating that the key is already present.

\begin{minted}{cpp}
bool XFastTrie::insert(Key key) {
    auto node = root;
    int level = 0;

    for (; level < bitWidth; ++level) {
        int bit = getBit(key, level);
        if (!node->children[bit]) break;
        node = node->children[bit];
    }

    if (level == bitWidth) {
        return false;
    }
\end{minted}

If the key is not already present, a series of helper methods is invoked. First, the predecessor and successor of the new key are located. Next, the remaining path down the trie is constructed based on the bits of the key. The new leaf is then linked between its predecessor and successor. Finally, jump pointers are updated by traversing up the trie, and the hash tables at each level are modified accordingly.

\begin{minted}{cpp}
    auto [predecessor, successor] = getInsertNeighbors(node, key, level);

    node->jump = nullptr;
    auto insertedNode = createPath(node, key, level);

    linkNeighbors(insertedNode, predecessor, successor);
    updateJumpPointers(insertedNode, key);
    updatePrefixLevels(key);

    return true;
}
\end{minted}

We can use the jump pointers directly to return the predecessor and successor of a key, given the lowest node matching a prefix of the key. If this node has no right child, the predecessor of the key will be the largest leaf in the node's left subtree, pointed to by the jump pointer; the successor will then be the predecessor's successor, or the minimum value in the trie if the predecessor does not exist. Similarly, if the node has no left child, the successor of the key will be the smallest leaf in the node's right subtree, pointed to by the jump pointer; the predecessor will then be the successor's predecessor, or the maximum value in the trie if the successor does not exist.

\begin{minted}{cpp}
std::pair<std::shared_ptr<Node>, std::shared_ptr<Node>> 
XFastTrie::getInsertNeighbors(
    std::shared_ptr<Node> node,
    Key key,
    int level
) const {
    int bit = getBit(key, level);
    if (bit == RIGHT) {
        auto pred = node->jump;
        auto succ = pred ? pred->linkedNodes[NEXT] : dummy->linkedNodes[NEXT];
        return {pred, succ};
    } else {
        auto succ = node->jump;
        auto pred = succ ? succ->linkedNodes[PREV] : dummy->linkedNodes[PREV];
        return {pred, succ};
    }
}
\end{minted}

The following utility functions complete the insertion by adding any missing nodes to the trie, linking the new leaf into the doubly linked list, and updating the hash tables at each level by traversing the full path to the leaf.

\begin{minted}{cpp}
std::shared_ptr<Node> XFastTrie::createPath(
    std::shared_ptr<Node> node, Key key, int startLevel) {
        for (int level = startLevel; level < bitWidth; ++level) {
            int bit = getBit(key, level);
            node->children[bit] = std::make_shared<Node>();
            node->children[bit]->parent = node;
            node = node->children[bit];
        }
        node->key = key;
        return node;
}

void XFastTrie::linkNeighbors(
    std::shared_ptr<Node> node,
    std::shared_ptr<Node> pred,
    std::shared_ptr<Node> succ
) {
    node->linkedNodes[PREV] = pred;
    node->linkedNodes[NEXT] = succ;
    if (pred) pred->linkedNodes[NEXT] = node;
    if (succ) succ->linkedNodes[PREV] = node;
}

void XFastTrie::updatePrefixLevels(Key key) {
    auto node = root;
    for (int i = 0; i <= bitWidth; ++i) {
        prefixLevels[i][getPrefix(key, i)] = node;
        if (i < bitWidth) {
            int bit = getBit(key, i);
            node = node->children[bit];
        }
    }
}
\end{minted}

We also need to update the jump pointers along the entire path. We do this by going from the leaf upward. At each node traversed, we make sure to maintain the invariant: if a node has no left child, it stores a pointer to the minimum leaf in its right subtree; if it has no right child, it stores a pointer to the maximum leaf in its left subtree.

\begin{minted}{cpp}
void XFastTrie::updateJumpPointers(std::shared_ptr<Node> node, Key key) {
    auto parent = node->parent;
    while (parent) {
        if ((parent->children[LEFT] == nullptr && 
            (!parent->jump || parent->jump->key > key)) ||
            (parent->children[RIGHT] == nullptr && 
            (!parent->jump || parent->jump->key < key))) {
            parent->jump = node;
        }
        parent = parent->parent;
    }
}
\end{minted}

\begin{lemma}
The average time complexity of the \texttt{push} operation on X-fast tries is \( O(\log M) \).
\end{lemma}

\begin{proof}
The complexity is dominated by a constant number of traversals over a path down the trie. Each such traversal requires \( O(\log M) \) time, as the height of the trie is \( \lceil \log_2 M \rceil \), which is the maximum length of a binary representation of an integer from the universe, and at each level we perform a constant amount of work. This is under the assumption that operations on the \texttt{prefixLevels} structure have average \( O(1) \) time complexity, which can be achieved e.g. using any data structure that supports perfect hashing.
\end{proof}

\subsection{Pop operation}
Like the \texttt{push} operation, the \texttt{pop} operation starts by traversing the trie along the path corresponding to the \texttt{key} to be removed. If the full path cannot be matched, it returns \texttt{false}, indicating the element is not present. Subsequently, several helper methods are invoked to remove the element from the linked list of leaves and to clean up both the trie path and the hash tables.

\begin{minted}{cpp}
bool XFastTrie::remove(Key key) {
    auto node = root;

    for (int level = 0; level < bitWidth; ++level) {
        int bit = getBit(key, level);
        if (!node->children[bit]) return false;
        node = node->children[bit];
    }

    cleanupPath(node, key);
    unlinkNode(node);

    return true;
}
\end{minted}

First, we remove the now unused nodes from both the trie and the hash tables at each level. If a node now has no children, we also clear its jump pointer.

\begin{minted}{cpp}
void XFastTrie::cleanupPath(std::shared_ptr<Node> node, Key key) {
    auto parent = node->parent;
    int level = bitWidth - 1;

    for (; level >= 0; --level) {
        int bit = getBit(key, level);

        parent->children[bit] = nullptr;
        prefixLevels[level + 1].erase(getPrefix(key, level + 1));
        if (!parent->children[1 - bit]) parent->jump = nullptr;

        if (parent->children[1 - bit]) break;
        parent = parent->parent;
    }
}
\end{minted}

Next, we update the jump pointers for the parent of the removed node and all ancestors above it that previously pointed to this node. If such an ancestor lacks a left child, its jump pointer is updated to the node’s successor, which is now the smallest key in the ancestor’s right subtree. Conversely, if the ancestor lacks a right child, we update its jump pointer to the node’s predecessor, now the largest key in the ancestor’s left subtree.

\begin{minted}{cpp}
if (parent) parent->jump = node;

for (; level >= 0; --level) {
    if (parent->jump == node) {
        if(!parent->children[LEFT]) {
            parent->jump = node->linkedNodes[NEXT];
        } else if(!parent->children[RIGHT]) {
            parent->jump = node->linkedNodes[PREV];
        }
    }
    parent = parent->parent;
}
\end{minted}

Finally, we remove the element from the linked list of leaves.

\begin{minted}{cpp}
void XFastTrie::unlinkNode(std::shared_ptr<Node> node) {
    if (node->linkedNodes[PREV]) {
        node->linkedNodes[PREV]->linkedNodes[NEXT] = node->linkedNodes[NEXT];
    }
    if (node->linkedNodes[NEXT]) {
        node->linkedNodes[NEXT]->linkedNodes[PREV] = node->linkedNodes[PREV];
    }
}
\end{minted}

The average time complexity of the \texttt{pop} operation on X-fast tries is \( O(\log M) \). We omit the analysis as it is closely analogous to the analysis for \texttt{push}.

\subsection{Predecessor operation}

While \texttt{pop} and \texttt{push} do not offer any advantage in terms of time complexity over standard binary tries, they maintain a structure that now allows us to elegantly perform \texttt{predecessor} queries in \( O(\log \log M) \) worst-case time. Given a key \(x\), \texttt{predecessor} returns \(x\) if it exists in the structure; otherwise, it returns the greatest element less than \(x\), or a null pointer if no such element exists.

Predecessors are found by a binary search over the levels of the trie to identify the longest matching prefix of the query key.

\begin{minted}{cpp}
std::shared_ptr<Node> XFastTrie::findPredecessorNode(Key key) const {
    int low = 0, high = bitWidth + 1;
    auto node = root;

    while (high - low > 1) {
        int mid = (low + high) / 2;
        auto it = prefixLevels[mid].find(getPrefix(key, mid));
        if (it == prefixLevels[mid].end()) {
            high = mid;
        } else {
            node = it->second;
            low = mid;
        }
    }
\end{minted}

After locating this prefix, the structure leverages jump pointers and predecessor/successor pointers from the linked list to identify the predecessor or successor of the corresponding node. If the first unmatched bit of the key is \(1\), we retrieve the largest key in the left subtree of the node via the jump pointer. Otherwise, we find the smallest key in the right subtree using jump pointers and then move one position backward in the linked list.

\begin{minted}{cpp}
    if (low == bitWidth && node->key.has_value()) {
        return node;
    }

    int dir = getBit(key, low);
    return (dir == RIGHT) ? 
        node->jump :
        (node->jump ? node->jump->linkedNodes[PREV] : nullptr);
}
\end{minted}

\begin{lemma}
The worst-case time complexity of the \texttt{predecessor} operation on X-fast tries is \( O(\log \log M) \).
\end{lemma}

\begin{proof}
The complexity is determined by the complexity of the binary search on the levels of the trie, as all other operations require constant time. This binary search looks for a value in a list of \( O(\log M) \) levels, and therefore it takes \( O(\log \log M) \) time.
\end{proof}
