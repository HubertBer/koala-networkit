\section{Mis1}

Let us start with a disclaimer that all algorithms presented below, bar the one presented in \Cref{sec:misfolding} are slower than the one published by Tarjan and Trojanowski. Despite that, they are interesting not only from the educational point of view but also as easy-to-implement points of reference for practical measurements -- as it is often the case that in practice simple algorithms may outperform the more convoluted ones with better worst-case guarantees.

Let us start with a definition of neighborhood.

\begin{defn}[neighborhood]
The set of all neighbors of a vertex $v$ of $G = (V, E)$, denoted by $N(v)$, is called the (open) \emph{neighborhood} of $v$.

By $N[v]$ we denote a closed neighborhood that additionally contains a vertex $v$. In mathematical terms, it is $N[v]=N(v) \cup \{v\}$.
\end{defn}
More generally, by $N^d(v)$ we denote the set of nodes at a distance $d$ from $v$. In particular, $N^1(v) = N(v)$.

\begin{lemma}
For any vertex $v\in G$, at least one vertex of $N(v)$ must belong to the maximum independent set of $G$.
\end{lemma}
\begin{proof}
    The proof is very straightforward. Let us assume otherwise that the set $I$ is a maximal independent set of $G$ and in $G$ exists $v$ such that no $N[v]$ belongs to $I$. We can add the vertex $v$ to $I$ and create a larger set than $I$ and also maximum. Hence, a contradiction.
\end{proof}

So, for any vertex $v$ we know that one of the $N[v]$ belongs to the solution. Using that observation, we can create a \textsc{Mis1} algorithm that will hopefully have a better computational complexity than $O^*(2^n)$. 

\begin{algorithm}
\caption{\textsc{Mis1}}\label{mis1}
\begin{algorithmic}[1]
\Require a graph $G=(V,E)$
\Ensure the maximum independent set of $G$
\Procedure{Mis1}{graph $G$}
    \If{$|V| = 0$}
        \State \Return $\varnothing$
    \EndIf
    \State choose arbitrary vertex $v$ of minimum degree in $G$
    \State $S \gets \{ \Call{Mis1}{G \setminus N[y]}\cup \{y\} \colon y \in N[v]\}$
    \State \Return any element of $S$ with the most elements    
\EndProcedure
\end{algorithmic}
\end{algorithm}

\subsection{Correctness}

The algorithm chooses a vertex $v$ and creates $N[v]$ subproblems. Each subproblem is another maximum independent set problem that will have to be solved. We know that some verticle $y$ of $N[v]$ has to be chosen to the independent set, and therefore, one of the subproblems plus the chosen vertex $y$ will give an optimal solution. Eventually, subproblems will be reduced to the size of $0$, and for them in fact, the empty set is the maximal independent set. After recursion finishes, it is going to return some maximal independent set.

Generally, we branch over some cases, reduce the original problem to new subproblems and then analyze them exactly as the original problem. This approach is known as \emph{Branch and Reduce}. 

\subsection{Computational complexity}

For \textsc{Mis1} algorithm, just once, we will do a full analysis of the complexity, including solving a recurrence. 

Let each node of the execution tree represent one call of the recursive function. We will be counting the number of nodes. Let $n$ be a number of vertices of graph $G$ and $T(k)$ be a function counting the maximum size of a branching tree for a graph $G$ consisting of $k$ vertices. We can obtain the following recurrence relation for each node:

$$
T(n) \leq 1 + \sum_{y\in N[v]} T(n - d(y) - 1)
$$

The first $1$ in the equation stands for one call before branching. The sum is over $N[v]$ because we branch over the neighbors of chosen $v$. And finally, $n - d(y) - 1$ is the size of the reduced subproblem. From the above recursive relation, we shall bound running time in $O^*$ terms.

From the choice of $v$ (we are choosing minimum degree vertex) we have $n-d(y)-1 \leq n - d(v) -1$ and from the monotonic property of $T$ we get the following:

\begin{equation*}
\begin{split}
T(n) & \leq 1 + \sum_{y\in N[v]} T(n - d(y) - 1) \\
 & \leq  1 + \sum_{y\in N[v]} T(n - d(v) - 1) \\
 & = 1 + T(n - d(v) - 1) \sum_{y\in N[v]} 1 \\
 & = 1 + T(n - d(v) - 1)(d(v) + 1)
\end{split}
\end{equation*}

Let $s = d(v) + 1$. We obtain the following.

$$
T(n) \leq 1 + s T(n - s)
$$

We can expand $T(n-s)$, and then use $T(0)=1$ and properties of geometric series
\begin{equation*}
\begin{split}
T(n) & \leq 1 + s T(n - s) \\
 & \leq 1 + s + s^2T(n - 2s) \\
 & \leq 1 + s + s^2 + \ldots + s^{n/s - 1} + s^{n/s} T(0) \\
 & = \frac{1-s^{n/s+1}}{1-s} \\
 & = O^*\left(s^{n/s}\right)
\end{split}
\end{equation*}

Since $s^{n/s} = (s^{1/s})^n$, it is enough to find the maximum of $f(s) = s^{1/s}$ for $s \in \mathbb{N}$. Indeed, $f$ will take the maximum value for $s=3$ which consequently leads to $O^*(3^{n/s}) \approx O^*(2^{0.5283n}) \approx O^*(1.4422^n)$ running time of our algorithm. This is certainly better than \textsc{NaiveMis}'s $O^*(2^n)$.