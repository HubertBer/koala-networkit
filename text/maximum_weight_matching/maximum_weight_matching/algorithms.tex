\section{Maximum matching algorithms}

The first polynomial algorithm for maximum weight matching in general graphs was given by Edmonds in~\cite{edmonds1965maximum}. It uses the primal-dual method and relies on his previous work on an algorithm for maximum cardinality matching~\cite{edmonds1965paths}. It is called the \emph{blossom algorithm} after blossoms, which are certain odd length cycles, a feature that notably does not appear in bipartite graphs and is a major source of complexity in algorithms for general graphs. The running time given by Edmonds was $O(n^2m)$. It was then independently improved to $O(n^3)$ by~\citeauthor*{lawler2001combinatorial}~\cite{lawler2001combinatorial} and~\citeauthor*{gabow1974implementation}~\cite{gabow1974implementation}. Many implementations of Edmonds' blossom algorithm followed with improved theoretical time complexity. The current best one due to~\cite{gabow1990data} runs in $O(nm + n^2 \log n)$ due to, which is in some sense optimal: the blossom algorithm works in $O(n)$ phases, a single of which can be used to sort $n$ numbers and therefore requires $\Omega(m + n \log n)$ time in a comparison-based model of computation. 

Other algorithms use an approach called scaling to solve the problem for graphs with integer edge weights. The weights are exposed one bit a time starting from the most significant one. In the $i$-th scale, the algorithm computes the optimal solution for weights $w_i$ consisting of $i$ significant bits of $w$. The solution is then used to more efficiently solve the next scale. The first algorithm based on this approach was presented in~\cite{gabow1984efficient} and runs in time $O(n^{3/4}m \log N)$. This was later improved to $O(m \sqrt{n\alpha(m, n) \log n} \log(nN))$ in~\cite{gabow1991faster} and lately $O(m \sqrt{n} \log(nN))$ in~\cite{duan2018scaling}. 

Additionally, there exists an algebraic randomized algorithm by~\cite{cygan2015algorithmic} running in $O(n^\omega N)$ with high probability for the matrix multiplication exponent $\omega$.

A summary of existing algorithms is presented in~\Cref{tab:mcm_alogrithms} and~\Cref{tab:mwm_alogrithms}.

\begin{table}
\centering
\begin{tabular}{clcl}
\toprule
\textbf{Year} & \textbf{Authors} & \textbf{Ref}                 & \textbf{Running time} \\
\midrule
1965 & Edmonds                   &\cite{edmonds1965paths}       & $O(n^2m)$ \\
1976 & Gabow                     &\cite{gabow1976efficient}     & $O(nm)$ or $O(n^3)$ \\
1976 & Lawler                    &\cite{lawler2001combinatorial}& $O(nm)$ or $O(n^3)$ \\
1980 & Micali \& Vazirani        &\cite{micali1980v}            & $O(\sqrt{n}m)$ \\
1991 & Gabow \& Tarjan           &\cite{micali1980v}            & $O(\sqrt{n}m)$ \\
1989 & Rabin \& Vazirani         &\cite{rabin1989maximum}       & $O(n^{\omega + 1})$ \\
2004 & Goldberg \& Karzanov      &\cite{goldberg2004maximum}    & $O(\sqrt{n}m/\kappa)$, $\kappa = \frac{\log n}{\log(n^2/m)}$ \\
2004 & Mucha \& Sankowski        &\cite{mucha2004maximum}       & $O(n^\omega)$ \\
\bottomrule
\end{tabular}
\caption{Maximum cardinality matching (\textsc{MCM}) algorithms.}\label{tab:mcm_alogrithms}
\end{table}

\begin{table}
\centering
\begin{tabular}{clcl}
\toprule
\textbf{Year} & \textbf{Authors} & \textbf{Ref}                 & \textbf{Running time} \\
\midrule
\textbf{1965} & \textbf{Edmonds}                   &\cite{edmonds1965maximum}     & $O(n^2m)$ \\
\textbf{1974} & \textbf{Gabow}                     &\cite{gabow1974implementation}& $O(n^3)$ \\
1976 & Lawler                    &\cite{lawler2001combinatorial}& $O(n^3)$ \\
\textbf{1985} & \textbf{Gabow}                     &\cite{gabow1985scaling}       & $O(n^{3/4}m \log N)$ \\
\textbf{1986} & \textbf{Galil, Micali \& Gabow}    &\cite{galil1986ev}            & $O(nm \log n)$ \\
1989 & Gabow, Galil \& Spencer   &\cite{gabow1989faster}        & $O(nm \log \log \log_{2 + \frac{m}{n}} + n^2\log n)$ \\
1990 & Gabow                     &\cite{gabow1990data}          & $O(nm + n^2 \log n)$ \\
1991 & Gabow \& Tarjan           &\cite{gabow1991faster}        & $O(m \sqrt{n \alpha(m, n) \log n} \log (nN))$ \\
2012 & Cygan, Gabow \& Sankowski &\cite{cygan2015algorithmic}   & $O(n^\omega N)$ \\
2018 & Duan, Pettie \& Su        &\cite{duan2018scaling}        & $O(m \sqrt{n} \log(nN))$ \\
\bottomrule
\end{tabular}
\caption{Maximum weight matching (\textsc{MWM}) algorithms. Implemented algorithms are marked in bold.}\label{tab:mwm_alogrithms}
\end{table}


We will take a closer look at a selection of maximum weight matching algorithms, namely the original $O(n^2m)$ blossom algorithm of Edmonds~\cite{edmonds1965maximum}, the $O(n^3)$ algorithm of Gabow~\cite{gabow1974implementation}, $O(nm \log n)$ algorithm of Galil, Micali and Gabow~\cite{galil1986ev} and the $O(n^{3/4}m \log N)$ scaling algorithm running of Gabow~\cite{gabow1985scaling}. We implement all of them in C++20 as part of the open-source KOALA NetworKit library~\cite{koala-networkit}. In~\autoref{chap:blossom} and~\autoref{chap:scaling} we describe how they work and how to implement them. In~\autoref{chap:results}, we present results of computational tests and compare the performance of our implementations.
